\documentclass[10pt,a4paper]{article}
\usepackage{graphicx} 
\usepackage{lastpage}
\usepackage{anysize}
\usepackage{fancyhdr}
\usepackage[german]{babel}
\usepackage[T1]{fontenc}
%\usepackage[utf8]{inputenc}
\usepackage{aeguill}
\usepackage{eurosym}
\usepackage{multicol}
\usepackage{amsmath} 
\usepackage{amssymb} 
\usepackage{amsthm}

%%% Format der Seite anpassen
\marginsize{2.5cm}{5cm}{0.85cm}{2.41cm}

%%% Header und Footer bauen
\pagestyle{fancy}
\renewcommand{\footrulewidth}{0.4pt}
\renewcommand{\headrulewidth}{0.4pt}
\renewcommand{\headheight}{2.0cm}
\renewcommand{\headsep}{0cm}
\setlength{\unitlength}{1mm}
\lhead{Matrikelnummer\\1234567\\}
\chead{\large{Einsendeaufgabe 1\\fuer den Kurs\\007}}
\rhead{\today\\\ \\\ }
\lfoot{\small{Telefon: +49 (123) 456 789 0\\
	e-Mail: max@mustermann.biz\\
	WWW: www.mustermann.biz}}
\cfoot{\thepage\ / \pageref{LastPage}}
\rfoot{\small{Max Mustermann\\Musterstr. 987\\56382 Musterstadt}}

\begin{document}
\setlength{\parindent}{0mm}
\setlength{\parskip}{6pt }

%%% Dokumentenkrams nur auf der ersten Seite

\hfill

%%% Dokumenteninhalt

Hier sollte dann irgendwas sinnvolles stehen... Oder zumindest irgendetwas, das den Anschein hat sinnvoll zu sein. 
Und wenn selbst das fehlschl"agt, dann sollte es die Korrektoren so weit verwirren, dass sie es nicht merken. ;)

\section{Beispiele}

Zum Beispiel kann man den Satz des Pythagoras mit \LaTeX\ gut darstellen:
\begin{equation}
	a^2 * b^2 = c^2
\end{equation}

Und auch ganz kranker Kram funktioniert:

\begin{equation} \left. \begin{aligned} \frac{dB}{dt} & = -\nabla \times E \\ \frac{dE}{dt} & = \nabla \times B - 4 \pi j \end{aligned} \right\}\qquad \text{Maxwellsche Gleichungen} \end{equation}

\section{Dokumentation}

Da in dieser Vorlage \AmS\ zum Einsatz kommt, gibt es sicherlich viele Dokus zum Verhalten, allerdings finde ich eine sehr interessant:

http://www.matheplanet.com/matheplanet/nuke/html/article.php?sid=740 (Oben den Link zur 24-Seiten-PDF, die m"ogen kein Hotlinking...)\\\small{(Hier habe ich auch die Maxwellsche Gleichungen entliehen ;))}

Ansonsten k"onnen alle normalen \LaTeX\ Anweisungen verwendet werden.


\end{document}
